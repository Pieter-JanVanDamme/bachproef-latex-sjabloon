%%=============================================================================
%% Methodologie
%%=============================================================================

\chapter{\IfLanguageName{dutch}{Methodologie}{Methodology}}
\label{ch:methodologie}

%% TODO: Hoe ben je te werk gegaan? Verdeel je onderzoek in grote fasen, en
%% licht in elke fase toe welke stappen je gevolgd hebt. Verantwoord waarom je
%% op deze manier te werk gegaan bent. Je moet kunnen aantonen dat je de best
%% mogelijke manier toegepast hebt om een antwoord te vinden op de
%% onderzoeksvraag.

Het onderzoek steunt op twee benen: een literatuurstudie (hoofdstuk \ref{ch:stand-van-zaken}) enerzijds, en een reeks vergelijkende performantietesten van de drie kandidaat testautomatisatiesystemen (hoofdstuk \ref{ch:corpus}) anderzijds.

De literatuurstudie geeft — naast een bondig doch volledig overzicht van de grondlagen van test automatisatie — een resumé van de functionaliteiten en eigenschappen van de drie frameworks die door Colruyt Group overwogen worden. Dit onderdeel heeft niet alleen een belangrijk aandeel in het beantwoorden van de onderzoeksvraag; het is tevens het eerste deel van het voorbereidend werk dat aan het praktische luik van het onderzoek vooraf gaat.

%% ``resumé'' wordt hier gebruikt in de betekenis ``beknopt overzicht''

Het andere deel van het voorbereidend werk voor de performantietesten is de analyse van de werking van de UAT (application under test), die te vinden is in bijlage \ref{ch:functionele-omschrijving}. Aan de hand van deze functionele omschrijving is het mogelijk representatieve test cases te kiezen voor die testen.

Zoals ook blijkt uit de literatuurstudie vereist een test automatisatiesysteem een weloverwogen architectuur die dient als fundament voor het systeem. De opbouw van deze basis eist een aanzienlijke investering van tijd en voldoende bekendheid met het framework. Om die reden wordt voor de performantietesten gebruik gemaakt van het bestaande werk van Thanuja Mudiyala (Test Automation Lead bij Colruyt Group Services) voor UFT en van Kenneth Aerens (Angular coach bij Colruyt Group Services) voor Cypress. De bestaande testen worden aangepast zodanig dat de stappen van elke test case precies dezelfde zijn. De implementatie van Protractor wordt volledig door de auteur gerealiseerd.

%% Cypress en UFT testen geschreven door anderen; raamwerk en simulatie van de peripherals is een enorme onderneming

De 5 gekozen test cases zijn als volgt:

\begin{enumerate}
    \item tekst
    \item tekst
    \item tekst
    \item tekst
    \item tekst
\end{enumerate}

.......... Wat meten we?

Tijdens de performantietesten wordt gebruik gemaakt van Windows \textsuperscript{\textregistered} Performance Monitor (ook gekend als ``PerfMon'') om de gekozen metrieken te loggen.

%% Gebruik van PerfMon:
%% 1. Run... perfmon.exe
%% 2. Ga naar Monitoring Tools > Performance Monitor en voeg counters toe, alsook instances van processen
%% 3. Rechtsklik op Performance Monitor New > Data Collector Set
%% 4. Rechtsklik op Data Collector Sets > User Defined > [YourDataCollectorSet] en klik Start om opname te starten
%% 5. Idem, maar Stop om te stoppen
%% 6. Open cmd.exe
%% 7. Gebruik cd om naar de directory van [YourDataCollectorSet] te gaan
%% 8. Zet .blg naar .csv om met commando 'relog MyMonitorLog.blg -f csv -o MyMonitorLog.csv'

............. De performantietesten geven eveneens inzicht in hoeveel (grondwerk, hoe moeilijk te schrijven)

Samen geven deze onderdelen een objectief beeld van de voor- en nadelen van de frameworks

