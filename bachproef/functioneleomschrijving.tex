%%=============================================================================
%% Inleiding
%%=============================================================================

\chapter{\IfLanguageName{dutch}{Functionele Omschrijving TAC2.0}{Functional Description TAC2.0}}
\label{ch:functionele-omschrijving}

Het checkoutsysteem van Colruyt Group is een complex gegeven dat bestaat uit een verscheidenheid aan individuele componenten en verwoven zit met zo goed als elk facet van de firma. Een grondige kennis van dit systeem, en in het bijzonder de werking van het systeem vanuit het perspectief van de kassabediende, is dan ook onontbeerlijk voor dit onderzoek.

\section{Algemeen}

In de filialen van Colruyt, Colruyt France, OKay, OKay City, Bio-Planet, Dreamland, Dreambaby, en Colruyt Group Spar worden klantenaankopen verwerkt middels de \textbf{TAC} (``tactile'') schermen.

[randapparaten]

[]

[filiaalserver draait applicaties buiten FVS, DB wordt gebruikt voor data die niet-FVS is... FVS is meer dan enkel de TAC]

\section{Kassaproces}

Colruyt Group is een sterk procesgerichte organisatie; 

Registreren en afhandelen van klantenaankoop < Verkopen van goederen < Verkoop

\section{Lorem Ipsum}

#