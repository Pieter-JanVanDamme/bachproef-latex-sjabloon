%%=============================================================================
%% Voorwoord
%%=============================================================================

\chapter*{\IfLanguageName{dutch}{Woord vooraf}{Preface}}
\label{ch:voorwoord}

%% TODO:
%% Het voorwoord is het enige deel van de bachelorproef waar je vanuit je
%% eigen standpunt (``ik-vorm'') mag schrijven. Je kan hier bv. motiveren
%% waarom jij het onderwerp wil bespreken.
%% Vergeet ook niet te bedanken wie je geholpen/gesteund/... heeft

Vier en een half jaar geleden schreef ik me in voor de opleiding bachelor in de toegepaste informatica in afstandsonderwijs. Ik was een iets jongere man en was vastberaden de koers van mijn leven om te slaan naar een andere windrichting. Het behalen van dit diploma — in combinatie met een fulltime job — zal één van mijn belangrijkste verwezenlijkingen zijn.

De reis is niet altijd verlopen zoals verwacht. De gezondheidscrisis die de wereld momenteel in zijn grip heeft, stond geenzins op de planning. Hetzelfde geldt voor mijn aanwerving bij Colruyt Group in het voorjaar van 2018. Hoewel mijn opleiding nog niet afgerond was, gaf de firma mij het vertrouwen om mijn carrière als ICT-professional te beginnen in een rol als functioneel analist.

Het onderwerp van mijn bachelorproef — een vergelijkende studie van kandidaat testing frameworks voor het nieuwe front-end systeem van Colruyt Group's kassasysteem — is niet toevallig gekozen. Als analist binnen de afdeling ``point of sale \& payment'' houdt ik me voornamelijk bezig met projecten met impact op het checkoutsysteem. Deze bachelorproef ligt dus in het verlengde van mijn specialisatie en laat me toe op een andere manier impact te hebben.

Ik stond er echter niet alleen voor. Ik wil in de eerste plaats Antonia Pierreux bedanken voor haar begeleiding tijdens de bachelorproef. Haar pragmatische en optimistische houding hebben me zeker geïnspireerd om deze scriptie tot een goed einde te brengen.

Koen Stockman, mijn co-promotor en applicatiemanager binnen mijn afdeling, Kenneth Aerens, onze Angular coach, en Thanuja Mudiyala, test lead voor Colruyt India, wil ik evenzeer bedanken. Gedurende mijn 2 jaar als analist bij Colruyt Group heb ik altijd graag met hen samengewerkt als collega's. Ook tijdens het uitwerken van deze bachelorproef hebben ze mij ondersteund met veel expertise en aanmoediging.

Ik wil ook Greet Dolvelde, sr. analist van onze afdeling, en Luc Devits, mijn directe chef, bedanken om mij de ademruimte die ik nodig had om mijn opleiding af te ronden te gunnen.

Vele anderen  hebben mij aangemoedigd tijdens mijn studie. Deze studie was een herculische taak die meer offers vroeg dan ik initieel had verwacht. Zonder de vele andere mensen in mijn leven — familie, vrienden en collega's — had ik dit nooit kunnen bereiken. Bij deze: dank jullie allemaal!

Ten slotte bedank ik ook u, de lezer, en hoop ik dat mijn werk u mag informeren en inspireren.

\begin{flushright}
Pieter-Jan Van Damme\linebreak
Lede, 20/08/2020
\end{flushright}