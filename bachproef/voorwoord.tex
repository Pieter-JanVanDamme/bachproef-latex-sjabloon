%%=============================================================================
%% Voorwoord
%%=============================================================================

\chapter*{\IfLanguageName{dutch}{Woord vooraf}{Preface}}
\label{ch:voorwoord}

%% TODO:
%% Het voorwoord is het enige deel van de bachelorproef waar je vanuit je
%% eigen standpunt (``ik-vorm'') mag schrijven. Je kan hier bv. motiveren
%% waarom jij het onderwerp wil bespreken.
%% Vergeet ook niet te bedanken wie je geholpen/gesteund/... heeft

Vier en een half jaar geleden schreef ik me in voor de opleiding bachelor in de toegepaste informatica in afstandsonderwijs. Ik was een iets jongere man en was vastberaden de koers van mijn leven om te slaan naar een andere windrichting. Het behalen van dit diploma — in combinatie met een fulltime job — zal één van mijn belangrijkste verwezenlijkingen zijn.

De reis is niet altijd verlopen zoals verwacht. De gezondheidscrisis die de wereld momenteel in zijn grip heeft, stond geenzins op de planning. Hetzelfde geldt voor mijn aanwerving bij Colruyt Group in het voorjaar van 2018. Hoewel mijn opleiding nog niet afgerond was, gaf de firma mij het vertrouwen om mijn carrière als ICT-professional te beginnen in een rol als functioneel analist.

...........................................................
.............

Koen Stockman en Kenneth Aerens, in de eerste plaats collega's

\begin{flushright}
Pieter-Jan Van Damme\linebreak
Lede, 20/08/2020
\end{flushright}